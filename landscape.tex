%%%%%%%%%%%%%%%%%%%%%%%%%%%%%%%%%%%%%%%%
% Class options                        %
%%%%%%%%%%%%%%%%%%%%%%%%%%%%%%%%%%%%%%%%
% Orientation:                         %
% portrait (default), landscape        %
%                                      %
% Paper size:                          %
% a0paper (default), a1paper, a2paper, %
% a3paper, a4paper, a5paper, a6paper   %
%                                      %
% Language:                            %
% english (default), spanish           %
%%%%%%%%%%%%%%%%%%%%%%%%%%%%%%%%%%%%%%%%
\documentclass[landscape]{uwposter}


\usepackage[absolute, overlay]{textpos}            % Figure placement
\usepackage{graphicx}
\usepackage[shortlabels]{enumitem}



\setlength{\TPHorizModule}{\paperwidth}
\setlength{\TPVertModule}{\paperheight}


%Global Background must be put in preamble
\usebackgroundtemplate%
{%
    \includegraphics[height=\paperheight,keepaspectratio,clip]{uwposter-images/background.png}%
}

\title{Geoscience Access and Inclusivity Network}
%\author
%{%
%    GAIN team%\inst{1}
%}
%% Optional:
%\institute
%{
%    \inst{1} Glaciology group, Earth and Space Sciences
%}
	
%% Or:
%\institute{Contact information}


%% Remove footline:
%\setbeamertemplate{footline}{}


\begin{document}


\begin{frame}
\begin{columns}[onlytextwidth]




\begin{column}{\textwidth/3 - 2cm}
\begin{block}{\alert{What is GAIN?}}
    A graduate student focused integrated field- and coursework program that breaks down field accessibility barriers, builds scientific capability, and encourages collaboration.

\end{block}

    \begin{block}{Mission}
       	The \textbf{\alert{G}}eoscience \textbf{\alert{A}}ccess and \textbf{\alert{I}}nclusivity \textbf{\alert{N}}etwork provides an avenue for graduate students to develop the field skills and experiences that are uniquely suited to their educational needs, background, scientific development, and future goals by:
         \begin{enumerate}[I.]
         \item
    	providing field-related education that will aid each student in developing their own coherent scientific narrative as they work towards their dissertation.
    	\item
	promoting equity and inclusivity in the geosciences community by promoting those who do not otherwise have access to field training, and by providing opportunities regardless of pre-existing knowledge.
    	\item
	cultivating mentorship and collaboration among graduate students, postdocs, faculty, and local scientists.
    	\item
	emphasizing justice, equity, diversity, and inclusivity in all activities, including, but not limited to, distributing grants, conducting coursework, and fostering mentorship.
	\end{enumerate}
	
    \end{block}
    
    \begin{block}{Pilot Program Philosophy}
    \textbf{\alert{GAIN}} is a pilot program with funding support for 3 years.
    Our timeline, and our mission inform our program philosophy, which values community building experience and approaches community discussions with humility. We are continuously seeking to improve \textbf{\alert{GAIN}} with respect to our mission and invite you to join us with your will to make fieldwork more accessible and collaborative.
    
       
    \end{block}
    
\end{column}




\begin{column}{\textwidth/3 - 2cm}
    \begin{block}{Funding Opportunity}
        One of the unique functions of \textbf{\alert{GAIN}} is to support and fund local field-related training.
        Rotating members of the program evaluate proposals submitted on a rolling basis through a blind peer-review based on a rubric that will prioritize 
        \begin{enumerate}[I.]
        \item
        Inclusivity, increasing field access,
        \item
        Professional development,
        \item
        Program sustainability,
        \item
        Intellectual merit.
        \end{enumerate}

        Students may apply for \textbf{small awards (\alert{<\$2500})} to support building field competency through participation in courses and existing field programs that cultivate field skills not supported by current coursework at the University of Washington.
        Students may also propose \textbf{larger field projects (\alert{>\$2500})} that pose new scientific questions addressed in collaboration with other graduate students and Pacific Northwest scientists.
        

    \end{block}
    
    \begin{block}{Safety and Planning}
    One of the principal barriers to fieldwork is coordination of logistics.
    As part of proposals, students will design a comprehensive equipment-use and field-safety plan to include:
        \begin{enumerate}[I.]
         \item
    	a thorough risk assessment of associated tasks,
    	\item
	an action plan of how the team will mitigate risk,
    	\item
	a check-in plan, and emergency response plan,
    	\item
	and an identification of risk-appropriate training and minimum training standards.
	\end{enumerate}
	
	By detailing an equipment-use and safety plan as part of the proposal, all members of the field team can plan intentionally around the skills they hope to develop.
	\textbf{\alert{GAIN}} reviewers can also make suggestions that seek to maximize positive outcomes and leverage community experience working in the Pacific Northwest.
    \end{block}


  \end{column}


\begin{column}{\textwidth/3 - 2cm}
    \begin{block}{Coursework and Mentorship}
        The aim of  \textbf{\alert{GAIN}} coursework is to increase competency in skills necessary to conduct effective fieldwork, and give students experience proposing field-science related questions. Certain elements of the curriculum will change with the instructor, but foundational material including proposal writing, critical theory of collaboration, and co-creation resources and research products would be taught as part of every offering. These goals mirror mentorship goals, which seek to expand mentoring networks, focusing particularly on mentors outside the ESS academic community. Through the course offerings and collaboration of proposal writeing \textbf{\alert{GAIN}}  will foster connection between students and faculty and external mentors, but also between peers and increase collaboration within the department.

    \end{block}



    \begin{alertblock}{How to get involved?}
        If you are interested in learning more about \textbf{\alert{GAIN}} as a prospective student member or as a potential mentor, field guide, or curriculum developer, fill out the QR encoded form and join our listserve!
        \begin{figure}
        \includegraphics[width=.4\linewidth]{uwposter-images/frame.png}
        \label{fig:qrcode}
        \end{figure}
        
        Even if you aren't sure if you can commit the time to \textbf{\alert{GAIN}} right now, any help connecting with capable mentors from your network or contributing to our scientific and outdoor equipment pool is appreciated!
        
    \end{alertblock}

\end{column}


\end{columns}


\begin{textblock}{0.76}(0.12, 0.90)
    \color{white}
    \sffamily
    \textbf{Acknolwedgements}
    \\
    Knut Christianson conceived the initial program. \textbf{\alert{GAIN}} is funded by the College of the Environment, the ESS department, and private department member donations.  
    Current \textbf{\alert{GAIN}} members include Ben Hills, Annika Horlings, Andrew Hoffman, Taryn Black, Jessica Badgeley, Gemma O'Connor, Lindsey Davidge, John Christian,  Max Stevens, Ed Waddington, TJ Fudge, Michelle Koutnik, Nick Holschuh, and Eric Steig.
    \end{textblock}


\end{frame}
\end{document}